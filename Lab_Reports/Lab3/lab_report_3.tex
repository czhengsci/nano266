%%%%%%%%%%%%%%%%%%%%%%%%%%%%%%%%%%%%%%%%%
% University/School Laboratory Report
% LaTeX Template
% Version 3.1 (25/3/14)
%
% This template has been downloaded from:
% http://www.LaTeXTemplates.com
%
% Original author:
% Linux and Unix Users Group at Virginia Tech Wiki 
% (https://vtluug.org/wiki/Example_LaTeX_chem_lab_report)
%
% License:
% CC BY-NC-SA 3.0 (http://creativecommons.org/licenses/by-nc-sa/3.0/)
%
%%%%%%%%%%%%%%%%%%%%%%%%%%%%%%%%%%%%%%%%%

%----------------------------------------------------------------------------------------
%	PACKAGES AND DOCUMENT CONFIGURATIONS
%----------------------------------------------------------------------------------------

\documentclass{article}

\usepackage[version=3]{mhchem} % Package for chemical equation typesetting
\usepackage{siunitx} % Provides the \SI{}{} and \si{} command for typesetting SI units
\usepackage{graphicx} % Required for the inclusion of images
\usepackage{natbib} % Required to change bibliography style to APA
\usepackage{amsmath} % Required for some math elements 
\usepackage{fancyhdr}
\usepackage[a4paper, margin=1in]{geometry}

\pagestyle{fancy}
\fancyhead{}
\lhead{\text{Nano 266}}
\rhead{\text{Chen Zheng A53048780}}

\setlength\parindent{0pt} % Removes all indentation from paragraphs

\renewcommand{\labelenumi}{\alph{enumi}.} % Make numbering in the enumerate environment by letter rather than number (e.g. section 6)

%\usepackage{times} % Uncomment to use the Times New Roman font

%----------------------------------------------------------------------------------------
%	DOCUMENT INFORMATION
%----------------------------------------------------------------------------------------

\title{Nano 266 \\ Quantum Mechanical Modeling of Materials \\ Lab 3} % Title

\author{Chen \textsc{Zheng}} % Author name

\date{\today} % Date for the report

\begin{document}

\maketitle % Insert the title, author and date

\begin{center}
\begin{tabular}{l r}
% Date Performed: & January 1, 2012 \\ % Date the experiment was performed
% Partners: & James Smith \\ % Partner names
% & Mary Smith \\
Instructor: & Professor Shyue Ping Ong % Instructor/supervisor
\end{tabular}
\end{center}

% If you wish to include an abstract, uncomment the lines below
% \begin{abstract}
% Abstract text
% \end{abstract}

%----------------------------------------------------------------------------------------
%	SECTION 1
%----------------------------------------------------------------------------------------

\section{Question 1}

The BCC-HCP transition in iron

\subsection{Ground State Energy}
% \label{definitions}
\begin{description}
\item[Ground state energy in BCC Structure:]
Using energy cutoff of 50 Ry, we do a ground state energy calculation of Fe with bcc structure with K-point setting range [7-13] and lattice parameter from 5.38 to 5.46 (Bohr), 
from the convergence test as showed in the Figure~\ref{fig:BCC_K_point_test}, indicate with K-point set to 11 (K-point number 112), the energy converge to within 1 meV/atom. 

\begin{figure}[h!]
\centering
\includegraphics[width=0.7\textwidth]{Q1_BCC_K_Points_test.png}
\caption{K point convergence test in BCC Fe}
\label{fig:BCC_K_point_test}
\end{figure} 

\item[Ground state energy in HCP Structure:]
Using energy cutoff of 50 Ry, we do a ground state energy calculation of Fe in HCP structure with K-point from 7 to 13, and lattice parameter from 4.75 to 4.84 (Bohr), from 
the convergence test as showd in the Figure~\ref{fig:HCP_1.72}, Figure~\ref{fig:HCP_1.73} and Figure~\ref{fig:HCP_1.74}, with K-point setting increases to above 9 (Unique K-point
number 72), the energy converge to within 1 meV/atom. 

\begin{figure}[h!]
\centering
\includegraphics[width=0.7\textwidth]{Q1_HCP_K_Point_172_test.png}
\caption{K point convergence test in HCP Fe, c/a ratio: 1.72}
\label{fig:HCP_1.72}
\end{figure} 

\begin{figure}[h!]
\centering
\includegraphics[width=0.7\textwidth]{Q1_HCP_K_Point_173_test.png}
\caption{K point convergence test in HCP Fe, c/a ratio: 1.73}
\label{fig:HCP_1.73}
\end{figure} 

\begin{figure}[h!]
\centering
\includegraphics[width=0.7\textwidth]{Q1_HCP_K_Point_174_test.png}
\caption{K point convergence test in HCP Fe, c/a ratio: 1.74}
\label{fig:HCP_1.74}
\end{figure}

\item[Lattice parameter optimization for BCC]
Using energy cutoff of 50 Ry, K-point set to 11, we do a lattice parameter optimization of BCC structure with lattice parameter between 5.2 to 5.55 with higher sampling
point density from 5.3 to 5.4. 
From the energy vs lattice relationship as showed in the Figure~\ref{fig:BCC_Lattic_Opt}, when lattice parameter equals to 5.37 a.u.~, we reached the lowest energy
per atom. 

\begin{figure}[h!]
\centering
\includegraphics[width=0.7\textwidth]{Q1_BCC_Lowest_Energy_Hull.png}
\caption{Lattice parameter optimization for BCC}
\label{fig:BCC_Lattic_Opt}
\end{figure}

\end{description} 

\subsection{Energy}
% \label{definitions}
\begin{description}
\item[$H_{2}$ DFT Energy]
The total DFT energy of $H_{2}$ is \SI{-1.176631}{Ha}, \SI{-32.01778}{\eV}. 
\end{description}

 
%----------------------------------------------------------------------------------------
%	SECTION 2
%----------------------------------------------------------------------------------------

\section{Question 2 }

Geometry optimization and energy of $N_{2}$

\subsection{Geometry}
% \label{definitions}
\begin{description}
\item[$H_{2}$ Bond Length]
From the geometry data, the x, y coordinates of both hydrogen atoms are zero, the Z coordinates are \SI{-0.55} and \SI{0.55}. Scaling the coordinates by \SI{1.889725} to convert to a.u., the final bond lenght of $H_{2}$ is \SI{1.1}{\angstrom}. 
\end{description} 

\subsection{Energy}
% \label{definitions}
\begin{description}
\item[$N_{2}$ DFT Energy]
The total DFT energy of $N_{2}$ is \SI{-109.502491}{Ha}, \SI{-2979.71597}{\eV}. 
\end{description}

%----------------------------------------------------------------------------------------
%	SECTION 3
%----------------------------------------------------------------------------------------

\section{Question 3}

Geometry optimization and energy of $NH_{3}$。

\subsection{Non Polarized}
% \label{definitions}
\begin{description}
\item[$NH_{3}$ Bond Length and Bond Angle]
From the geometry data, the x, y, z coordinates of both hydrogen atoms and N atoms, the bond length between N and H is \SI{1.00591}. Scaling the coordinates by \SI{1.889725} to convert to a.u., the final N-H bond length of is \SI{1.9009}{\angstrom}. The bond angle is \SI{116.18}{\degree}.
\item[$NH_{3}$ Total DFT Energy]
Total DFT energy of $NH_{3}$ is \SI{-56.55168}{Ha}, \SI{-1538.850329}{\eV}.	
\end{description} 

\subsection{Polarized}
% \label{definitions}
\begin{description}
\item[$NH_{3}$ Bond Length and Bond Angle]
From the geometry data, the x, y, z coordinates of both hydrogen atoms and N atoms, the bond length between N and H is \SI{1.01801}. Scaling the coordinates by \SI{1.889725} to convert to a.u., the final N-H bond length of is \SI{1.018}{\angstrom}. The bond angle is \SI{107.68}{\degree}.
\item[$NH_{3}$ Total DFT Energy]
Total DFT energy of $NH_{3}$ is \SI{-56.55168}{Ha}, \SI{-1538.850329}{\eV}.
\end{description}


%----------------------------------------------------------------------------------------
%	SECTION 4
%----------------------------------------------------------------------------------------

\section{Formation enthalpy of $NH_{3}$}

To get the right formation enethalpy of $NH_{3}$, we first reset the basis set of $H_{2}$ and $N_{2}$ in Q1 and Q2 with polarization functions. After rerun $H_{2}$ and $N_{2}$ calculations, we get the thermal correction and DFT energy of each parts as table show below.

\begin{tabular}{lllr}
Compound & Energy (Ha) & Correction (kcal/mol) & Enthalpy H (kcal/mol) \\
$H_{2}$ & \SI{-1.17663} & \SI{8.436} & -729.903782 \\
$N_{2}$ & \SI{-109.56056} & \SI{5.580} &  -68744.00271\\ 
$NH_{3}$ & \SI{-56.57363} & \SI{24.042} &  -35476.07972\\ 
\end{tabular}

Given by the formula: $0.5 N_{2} + 1.5 H_{2} > NH_{3}$, the calculated formation enthalpy is -38.59 kJ/mol.\cite{NISTNH3}, which is not far from the experimental data of NIST source: http://cccbdb.nist.gov/, -45.95 kJ/mol. 

%----------------------------------------------------------------------------------------
%	SECTION 5
%----------------------------------------------------------------------------------------

\section{Effect of Functional Choice and Basis Set}

In this section, we enumerated all 36 differenct conditions with different functional choice and basis set. Only basis set with polarization function are used.We use HF, PBE and B3LYP(B3) funtionals with 6-31G* and 6-311G* basis set, for each compound calculation, there are two stages, i.e. Geometry optimization stage and DFT energy stage. The experimental formation enthalpy for $NH_{3}$ is $ -45.95 \pm0.35 kJ/mol$ from NIST website.\cite{NISTNH3}. The bond length of N-H is $1.017\mathring{A}$ based on \cite{Wiki} data. The bond angle is $107.8^{\circ}$ based on wiki \cite{Wiki} data also. 

\subsection{Overview}
From the Figure~\ref{fig:overviewdev}, we could find that with difference functional set, the deviation of formation enthalpy various a lot. But the combination of different functional set provides us with enough data point that fall in the accepted deviation band, the band width is $\pm2\%$. The best functional set combination is using PBE function for both stages calcuation, using 6-31+G* as basis set for geometry stage optimization and using 6-311+G* for DFT energy calculation with calculated formation energy -46.18 kJ/mol, with only 0.5\% deviation from experimental value. 


% \begin{figure}[h]
% \includegraphics[width=1\textwidth]{Functional_used_Devi_overview.png}
% \caption{Overview of Formation Energy Deviation}
% \label{fig:overviewdev}
% \end{figure} 


\subsection{B3LYP Function and Effect}
If we use B3LYP functional as first stage function, to minimize the deviation from experimental data, we should keep using B3LYP as second stage function from Figure~\ref{fig:B3}. 
% \begin{figure}[ht]
% \includegraphics[width=1\textwidth]{B3LYP_Functional_1_Dev.png}
% \caption{B3 as geometry optimization stage functional}
% \label{fig:B3}
% \end{figure} 

\subsection{HF Function and Effect}
If we use HFexch functional as first stage function, to minimize the deviation from experimental data, B3LYP should still be considered as second stage function Figure~\ref{fig:HF}. 
% \begin{figure}[ht]
% \includegraphics[width=1\textwidth]{HF_Functional_1_Dev.png}
% \caption{HF as geometry optimization stage functional}
% \label{fig:HF}
% \end{figure} 

\subsection{PBE Function and Effect}
If we use PBE functional as first stage function, to minimize the deviation from experimental data, B3LYP should still be considered as second stage function Figure~\ref{fig:PBE}. 
% \begin{figure}[ht]
% \includegraphics[width=1\textwidth]{PBE_Functional_1_Dev.png}
% \caption{PBE as geometry optimization stage functional}
% \label{fig:PBE}
% \end{figure} 

\subsection{Function and Geometry Angle Prediction}
Using different functional, the N-H bond predicted are not much different and all pretty close to the data we found from online source wiki \cite{Wiki}, as show in Figure~\ref{fig:Geometry}. The blue horizontal line is the experimental angle data we found from wiki. 
% \begin{figure}[ht]
% \includegraphics[width=1\textwidth]{Geometry.png}
% \caption{N-H angle predicted by various functional set}
% \label{fig:Geometry}
% \end{figure}

\subsection{Function and Bond Length Prediction}
Using different functional, the N-H bond length predicted are not much different and all pretty close to the data we found from online source wiki \cite{Wiki}, as show in Figure~\ref{fig:Bond Length}. The blue horizontal line is the experimental angle data we found from wiki. The B3LYP functional use 6-31+G* (optimization stage) and 6-311+G* (energy stage) gives the best geometry prediction with highest accuracy.
% \begin{figure}[ht]
% \includegraphics[width=1\textwidth]{Bond_Length.png}
% \caption{N-H bond length predicted by various functional set}
% \label{fig:Bond Length}
% \end{figure}

\subsection{CPU wall time and effectiveness}
If we could accept $2\%$ as deviation, from Figure~\ref{fig:walltime}, using HF functional for geometry optimization and PBE for DFT energy calculation might be the most efficient calculation strategy with relatively good accuracy within 2\%. 
% \begin{figure}[ht]
% \includegraphics[width=1\textwidth]{walltime.png}
% \caption{CPU walltime summary within acceptable deviation}
% \label{fig:walltime}
% \end{figure}

\subsection{Basis set effect}
In this section, we will investigate the effect of 6-31+G* and 6-311+G* basis set in second stage (DFT energy calculation) calculation, from the below bar chart~\ref{fig:6_31}, using 6-311+G* as second stage function, the predicted formation energy is more close to experimental value with decreased standard deviation also. 

\begin{tabular}{ll}
Row content & Formation Enthalpy(kJ/mol) \\
6-31+G* mean value & -39.640336 \\
6-31+G* standard deviation & 6.651545 \\ 
6-311+G* mean value & -38.815557\\
6-311+G* standard deviation & 5.669244 \\ 
\end{tabular}

\begin{tabular}{llll}
Row content & Wall Time (s) & N-H Angle & N-H Length\\
6-31+G* mean value & 26.588889 & 107.623333 & 1.015263 \\
6-31+G* standard deviation & 13.591214 &   0.456645 & 0.010470\\ 
6-311+G* mean value & 26.588889 & 107.623333 & 1.015263 \\
6-311+G* standard deviation & 13.282392 &  0.456645 & 0.010470\\ 
\end{tabular}

% \begin{figure}[ht]
% \includegraphics[width=1.2\textwidth]{Basis_Set_6_31_1st.png}
% \caption{CPU walltime summary within acceptable deviation}
% \label{fig:6_31}
% \end{figure}


%----------------------------------------------------------------------------------------
%	SECTION 6
%----------------------------------------------------------------------------------------
\section{Question 6}

For the reaction to happen, we need to break $N-N$ triple bond, the atomic energy of each N is -54.6 Ha. Compared with energy of N2, -109.53 Ha, the dissociation energy of N2 is 0.33 Ha = 8.98 ev, 207.08 kcal/mol. This is also the reaction barrier of reaction. 


%----------------------------------------------------------------------------------------
%	BIBLIOGRAPHY
%----------------------------------------------------------------------------------------

\bibliographystyle{apalike}

\bibliography{sample}

%----------------------------------------------------------------------------------------


\end{document}